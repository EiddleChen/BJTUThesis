%# -*- coding: utf-8-unix -*-
%%==================================================
%% conclusion.tex for BJTUThesis
%% Encoding: UTF-8
%%==================================================

\begin{summary}

这里是全文总结内容。

2015年2月28日,中央在北京召开全国精神文明建设工作表彰暨学雷锋志愿服务大会,公布全国文明城市(区)、文明村镇、文明单位名单。北京交通大学荣获全国文明单位称号。         

全国文明单位这一荣誉是对交大人始终高度重视文明文化工作的肯定,是对交大长期以来文明创建工作成绩的褒奖。在学校党委、文明委的领导下,交大坚持将文明创建工作纳入学校建设世界一流大学的工作中,全体师生医护员工群策群力、积极开拓,落实国家和北京市有关文明创建的各项要求,以改革创新、科学发展为主线,以质量提升为目标,聚焦文明创建工作出现的重点和难点,优化文明创建工作机制,传播学校良好形象,提升社会美誉度,显著增强学校软实力。2007至2012年间,北京交大连续三届荣获“北京市文明单位”称号,成为创建全国文明单位的新起点。         

北京交大自启动争创全国文明单位工作以来,凝魂聚气、改革创新,积极培育和践行社会主义核心价值观。坚持统筹兼顾、多措并举,将争创全国文明单位与学校各项中心工作紧密结合,着力构建学校文明创建新格局,不断提升师生医护员工文明素养,以“冲击世界一流大学汇聚强大精神动力”为指导思想,以“聚焦改革、多元推进、以评促建、丰富内涵、彰显特色”为工作原则,并由全体校领导群策领衔“党的建设深化、思想教育深入、办学成绩显著、大学文化丰富、校园环境优化、社会责任担当”六大板块共28项重点突破工作,全面展现近年来交大文明创建工作的全貌和成就。         

进入新阶段,学校将继续开拓文明创建工作新格局,不断深化工作理念和工作实践,创新工作载体、丰富活动内涵、凸显创建成效,积极服务于学校各项中心工作和改革发展的大局面,在上级党委、文明委的关心下,在学校党委的直接领导下,与时俱进、开拓创新,为深化内涵建设、加快建成世界一流大学、推动国家进步和社会发展而努力奋斗!       

北京交通大学医学院附属仁济医院也获得全国文明单位称号。      

\end{summary}

